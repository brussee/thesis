% related work = related scientific articles
% NOT engineering (tsap, at3, tribler-mobile (ppsp-twitter))
% https://scholar.google.nl/scholar?hl=en&q=video+on+demand+android&btnG=&as_sdt=1%2C5&as_sdtp=


\chapter{Introduction}
% 3 to 4 pages with screenshots
% related work in citations in between text

In the era of social media people on earth are more connected then ever before.
%stats

However not every place on earth has an uncensored Internet connection, or has one that can be shut down with the push of a button.
%examples

Although smart phones have brought the Internet into the hands of people, this mobile device is not capable of overturning the power of the Internet-kill-switch, yet.
This is about to change as the self-organising video-on-demand platform Tribler is going to make the jump to mobile devices.
% relevance
A big part of social media is video sharing and streaming.
%stats
There is a great interest in this considering the amount of websites and apps available for streaming video-on-demand services.
%list apps?
Huge video streaming providers like Youtube, Twitch, Periscope, etc. currently dominate the market.
%stats
The problem with those is that none of them are server-less and do not  provide anonymity in any shape or form.
%why


\section{Unavailable infrastructure}

Internet kill-switches

Overloaded

Physically destroyed

%first point
With servers central to their design they create a single point of failure, even in a decentralized set-up.
Several natural disasters have taken out the necessary infrastructure on numerous occasions for a prolonged period of time.
%examples
Especially in situations like these, people need to communicate and coordinate their efforts to restore safety.
Social media has played a major role in recent calamities when people could mark themselves as safe, effectively broadcasting that information to all their family and friends on social media, instead of contacting them one by one or not at all due to congestion in the communication channels.
So the advantages are obvious, and the vulnerability of central elements underlying current social media too.


\section{Untrusted infrastructure}

Censorship

Partly disconnected

%second point
The lack of anonymity becomes a problem when the users privacy is being invaded.
Revealing personal information can be deduced from search queries for example, or associations on social platforms.
When this information can be used for targeted advertising it becomes very valuable, and creates an incentive for the parties that have access to this information to sell it to third parties.
In fact the business model of social media appears to be serving targeted advertisements to its users on behalf of third parties.
What's even worse is social media integrated into regular websites to de-anonymize and track the whereabouts of users even outside of the social media realm.
Whenever users lose control over their privacy it becomes a serious problem.


\chapter{Problem description}
% difficulty or opportunity
% develop software / purchase software / develop non-software solution
% FIRST and scientific, then figures and then filling in the holes


% http://www.internetsociety.org/articles/moving-toward-censorship-free-internet
% https://tools.ietf.org/html/draft-pouwelse-censorfree-scenarios-01#page-4
% http://www.un.org/en/universal-declaration-human-rights/

% Internet censorship
The Internet has given the opportunity to communicate freely on a global scale.
Connecting to it and crossing state borders on-line does not require approval of any governmental body.
However, central Internet exchange infrastructures are vulnerable to large scale abuse beyond total monitoring and filtering.
As such, not everyone has unrestricted access to the Internet due to surveillance and censorship.
A significant part of today's Internet users is affected by these attempts to hide or distort reality.
This interference directly affects the universal right to freedom of opinion and expression as stated in article 19 of the Universal Declaration of Human Rights (UDHR).
The sophistication of censorship techniques is pushed forward by the drive to stay ahead of attempts trying to circumvent it.

% Privacy
Pervasive monitoring of digital citizens by Internet providers on behalf of governments to enforce censorship laws raises severe privacy concerns.
Even the business model of social media companies directly conflicts with user privacy.
Targeted advertising requires the very information of high quality (accurate and current) users tend to share with their friends on-line.
When this information is shared with other parties outside of the specific social media website, possibly unknowingly to the user, it effectively becomes a privacy leak.
Subsequently users can be confronted with their information being misused in various ways beyond their control.
This lack of control over your own privacy can lead to arbitrary interference as defined in UDHR article 12. %ref, example, human rights watch, nelie kroes, etc.
Integration of social media on regular websites aggravates this problem.
Every page-view and click on social media enabled websites becomes traceable to an individual, directly benefiting the business model of targeted advertisements

% Internet censorship 2
The incentive to de-anonymize the user, not only causes a lack of privacy, but also a potential lack of freedom of expression, as it hands key information to the censor: who is expressing dissent and who is associated with this person on-line.
Cyber suppression has become a reality when you no longer can be associated with opinion-makers or foreign journalists on-line.

\section{Social journalism}
Social media has been a driving force in the Arab Spring 
Multi-media 
Camera phones
People have used social media to reach 20 million 


 excludes large portions of the global dialog on social media.


sniffing, blocking, filtering, shutdown


fragmentation of efforts for freedom

%everybody should

Some parts of the Internet

Increasingly though, Internet traffic is put under surveillance and obfuscation techniques are targeted by restrictions.

Internet censorship directly affects millions of people


\section{Adversary model}



\begin{figure}[h]
	\centering
	\includegraphics[width=\textwidth]{viral_spreading}
	\caption{Any device can spread to other devices wirelessly.
		Only one device has to travel or connect outside the offline region to make the content connectable to the Internet.}
	\label{fig:viral_spreading}
\end{figure}



%scope
To ensure that no controlling party can exercise censorship we distribute authority over all users, creating an \emph{autonomous} system.
If all information is located in one or a few places, the parties in charge of that location will still have control over it, so we must distribute information over all users, creating a \emph{communication} system.
Then if all users want to use this system to share, order and appreciate each others information, in other words the essence of social media: social interaction, with everyone being able to interact in the same way, we need to  distribute functionality over all users, creating a \emph{cooperation} system.
Fully distributed systems capture these characteristics. %move to solution?
Without any central component in the system it is no longer susceptible to censorship without everyone participating.

Peer-to-peer communication technology is essential for a server-less distributed system.
Mobile devices typically do not require infrastructure to exchange information, like those equipped with Bluetooth or capable of ad hoc Wi-Fi.
Smart phones are ubiquitous everywhere in the world and used to access social media and retrieve information from the Internet.
Fortuitously these are also the type of mobile devices that can communicate peer-to-peer.

\section{Thesis definition}
The main question thus becomes:
How to create a \textbf{self-organising} \textbf{video-on-demand} platform that is \textbf{attack-resilient} and can \textbf{operate autonomously} on a \textbf{mobile device}?

Self-organising in the sense that the platform coordinates the exchange of videos and meta-data fully automatically.

Video-on-demand in the sense that users can simply click and play videos in a streaming fashion, so without waiting for the entire video to be present on the device.

Attack-resilient in the sense that:
First: censorship does not have an effect if the majority of users does not cooperate with the censor.
Second: the privacy of users remains protected while they actively participate on the platform
Third: no network infrastructure required for viral spreading of the entire video platform.

Autonomous operation in the sense that users do not have to manage any files or configuration manually at all to be active on the platform.

Mobile device in the sense that it is low-powered and portable including the network interface and power supply.

These properties will ensure social media with resilience against Internet kill switches, natural disasters and censorship.



\section{Research Limitations} %project scope
% broad / narrow scope, scope of other system()s)
% sub-problems
% ultimate high-level goal
Software development / technical aspect only
Not policy making, organisational perspective, decision making, normative, ethical,
Time limit of 9 months



%LEESWIJZER











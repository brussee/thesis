\chapter{Problem description}
% FIRST and scientific, then figures and then filling in the holes


% http://www.internetsociety.org/articles/moving-toward-censorship-free-internet
% https://tools.ietf.org/html/draft-pouwelse-censorfree-scenarios-01#page-4
% http://www.un.org/en/universal-declaration-human-rights/


The business model of social media directly conflicts with user privacy.
Targeted advertising becomes trivial when the user profile is provided by the user itself with accurate and current information.
When the information is shared with parties outside of the specific social media website it effectively becomes a privacy leak.
Subsequently users will be confronted with their information being misused in various ways, beyond their control.
This lack of control over your own privacy is a problem and can lead to arbitrary interference, possibly even unknowingly.
The Universal Declaration of Human Rights (UDHR) declares in article 12 that everyone has the right to protection against such interference.

\begin{displayquote}
No one shall be subjected to arbitrary interference with his privacy, family, home or correspondence, nor to attacks upon his honour and reputation. Everyone has the right to the protection of the law against such interference or attacks.
(Article 12. UDHR)
\end{displayquote}

Integration of social media on regular websites can easily de-anonymize the visiting user, directly benefiting the business model of targeted advertisements, aggravating this problem.
This incentive not only causes a lack of privacy, but also, paradoxically, a potential lack of freedom of expression.
Every interest and opinion expressed on social media and beyond, whether that is by clicking on anything or publishing videos, is traceable to an individual due to this integration and business model.
If the controlling party of any website connected to social media wants to silence a certain opinion, it can use this connection to trace the individual and any like minded people connected by social interaction.
Particularly on social media do people exercise their right to freedom of opinion and expression.
The UDHR declares in article 19 that everyone has the right to do so without any interference whatsoever.

\begin{displayquote}
Everyone has the right to freedom of opinion and expression; this right includes freedom to hold opinions without interference and to seek, receive and impart information and ideas through any media and regardless of frontiers.
(Article 19. UDHR)
\end{displayquote}

To ensure that no controlling party can exercise censorship we \textbf{distribute authority} over all users, creating an \emph{autonomous} system.
If all information is located in one or a few places, the parties in charge of that location will still have control over it, so we must \textbf{distribute information} over all users, creating a \emph{communication} system.
Then if all users want to use this system to share, order and appreciate each others information, in other words the essence of social media: social interaction, with everyone being able to interact in the same way, we need to  \textbf{distribute functionality} over all users, creating a \emph{cooperation} system.
This captures the characteristics of a distributed system.
Without any central component in the system it is no longer susceptible to censorship without everyone participating.

Peer-to-peer communication technology is essential for a server-less distributed system.
Mobile devices typically do not require infrastructure to exchange information, like those equipped with Bluetooth or capable of ad hoc Wi-Fi.
Smart phones are ubiquitous everywhere in the world and used to access social media and retrieve information from the Internet.
Fortuitously these are also the type of mobile devices that can communicate peer-to-peer.



Automonous operation

Social media on phones

You want to express freedom of expression with that all the time % Normative, assumption of this thesis



Existing apps use central server design
Vulnarable for Internet kill switches and censorship

Offline viral spreading image (hacking lab)


How I did the work
What you have done told by what you did not do.
No answer completely, with tests that I did, but what I tested says this.... and makes it reasonable to say probably yes.


Image of what I got in the end and what I got in the beginning next to each other.


Is the main question legitimate?


What is the problem this VOD platform solves?


Regulatory perspective, out of scope
Ethical, out of scope


What do we want? VOD
What is already there? p4a
What is the gap? my work

How can we develop an autonomous and anonymous VOD patform with existing python code base that is ()user friendly /) working on a mobile platform and which is resistant to Internet kill-switches?

autonomous, not dependent on outside stuff
user friendly, power draw, special permissions

multicoin, mobile data, nodes/hops



Imperative programming has limitations
Reactive fits mobile paradigm better


Iterations, waterfall?
Delft system approach:
Systeem -> Functional Design -> alternatives -> choose ->Prototype implementation
																									 |
|																									|
------			 validation				 <----------------------------------------

KPI 's, speed, availability, ...
or
App is done
Now: multi-criteria analysis of all other apps and mine



What is VOD?
What does working on mobile phone mean?
What are the existing modules we re-use?
What is missing? (gap)
What are the possible solutions?
Is this prototype a verified solution to main problem desc.?


this is depth first analysis now.

NOT product description

\section{Scope}
No routing, no blutooth p2p, mesh networking: that is done by Serval. Out of scope of this work. Add as enhancement later.



maybe problem description in an image, like system diagram



%%
Intro per chapter, refer to problem definition each time, where are we now, what now (for reader)
Why this chapter
Paragraphs
Conclusion of contents of this chapter
Link to next chapter



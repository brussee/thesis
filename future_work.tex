\section{Future work}
\label{ch:future_work}

The contributions of this work enable future research into Tribler functionality on mobile devices in terms of the intended use-cases.
And any enables any other future research fully geared towards mobile devices.
Like for example a large scale experiment with various degrees of a powerful censor or a live deployment of Tribler mobile in areas with restricted Internet access to evaluate the effectiveness.



shared keychain, credit minging laptop, server, mobile..


NFC is used to share a channel id and to automatically setup a Bluetooth file transfer.
In the future the same method can be used to exchange bootstrap peers for the Tribler network.
The peer-to-peer Bluetooth file transfer could be made much faster if the WiFi Direct would be used.


Thumbnails retrieval and distribution now that video is so important on mobile, could be added on top of groundwork by me.


credit mining on device A, usage on device B


comparative analysis

Automate ad hoc WiFi direct like nfc-bluetooth!

cutting long running methods in smaller pieces, so multi-threading is smoother (only discovered now, earlier: why is gui hanging?)
release GIL in native C code, mainly for heavy crypto tasks
streaming API for big responses (dumping all known channels from db)

knippen voor twisted is gewenst, nu pas zichtbaar weakness in Tribler, nuttig om te verbeteren. (conclusion)


iOS with QtPython


% Research Limitations
% Project scope
% broad / narrow scope, scope of other system()s)
% sub-problems
% ultimate high-level goal
Software development / technical aspect only
Not policy making, organisational perspective, decision making, normative, ethical,
Time limit of 9 months


refereren aan eigen werk in verbergen app en mensen beschermen op bepaalde manier
combine with selfcompileapp
%mooie afsluiting, boog.
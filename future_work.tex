\section{Future work}\label{sec:future_work}
% Research
This work enables a new direction of research with Tribler fully geared towards mobile devices.
% future phd's and new agenda thanks to this work?
Future research can evaluate if, perhaps in the wild, smart-phones with Tribler can defeat the powerful adversary as described in Section \ref{sec:adversary_model}.
One can setup a large scale experiment with various degrees of powerful censors, to evaluate Tribler in the intended use-cases.
Or a live deployment of Tribler on mobile devices in areas with restricted Internet access, to evaluate the effectiveness of the .

And study how the viral spreading of eyewitness content behaves in the real world.
The influence on anonymity of local crowd effects on the onion routing protocol can be looked into, for example as well.

% Implementation
The modularity of the current implementation of the design enables Tribler to be easily ported to other platforms.
After our implementation was finished the standard Android integrated development environment (IDE) Android Studio started to officially support the Android NDK.
So, we can now move from the experimental alpha release to the stable Gradle plugin.
This enables Gradle to cross-compile the C/C++ libraries instead of the build tool-chain of Python-for-Android (P4A).
%However, the Python bindings and Tribler's source code still require Setuptools to be run as part of the build tool-chain.
This also makes it easier to replace P4A with for example Qt for Android, which gained support for Android services after our implementation was already finished.
The new desktop GUI of Tribler is built with Qt, and it would be nice to re-use code.





No routing, no blutooth p2p, mesh networking: that is done by Serval. Out of scope of this work. Add as enhancement later.
We won't modify the networking capabilities based on the opportunities because that would warrant a dedicated study.


In API version 18 support is added for Bluetooth Low Energy (LE), WiFi scan only mode, key store for app-private keys, hardware credential storage and automated UI testing.


The portability of our design and implementation allows it to be used in other applications like smart-tv's as well.



shared keychain, credit minging laptop, server, mobile..


NFC is used to share a channel id and to automatically setup a Bluetooth file transfer.
In the future the same method can be used to exchange bootstrap peers for the Tribler network. Or integrate into apk just before sending via Bluetooth.
The peer-to-peer Bluetooth file transfer could be made much faster if the WiFi Direct would be used.


Thumbnails retrieval and distribution now that video is so important on mobile, could be added on top of groundwork by me.


credit mining on device A, usage on device B


comparative analysis

Automate ad hoc WiFi direct like nfc-bluetooth!

cutting long running methods in smaller pieces, so multi-threading is smoother (only discovered now, earlier: why is gui hanging?)
release GIL in native C code, mainly for heavy crypto tasks
streaming API for big responses (dumping all known channels from db)

knippen voor twisted is gewenst, nu pas zichtbaar weakness in Tribler, nuttig om te verbeteren. (conclusion)


iOS with QtPython


% Research Limitations
% Project scope
% broad / narrow scope, scope of other system()s)
% sub-problems
% ultimate high-level goal
Software development / technical aspect only
Not policy making, organisational perspective, decision making, normative, ethical,
Time limit of 9 months


refereren aan eigen werk in verbergen app en mensen beschermen op bepaalde manier
combine with selfcompileapp
%mooie afsluiting, boog.
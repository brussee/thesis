\chapter{Conclusions and future work}\label{ch:conclusions}
% After the performance analysis of our implementation in the previous chapter, we now come to the conclusions with regard to the research questions.
% Finally, in Section \ref{sec:future_work} we will describe possible avenues for future research based on this work.

All Tribler functionality runs on mobile devices with our implementation.
The code is open source \cite{github_tribler_repo} and does not rely on any proprietary app store.
This is the first successful attempt, to our knowledge, of creating a self-organizing video-on-demand platform that is attack-resilient and can operate autonomously on a mobile device.
Millions of people that own a smartphone, and not a computer, can now benefit from Tribler's privacy enhancing functionality.


\section{How feasible is it to run all Tribler functionality on mobile devices?} % to defeat or mitigate large scale monitoring and censorship?
We proposed an Android implementation that fulfills are design requirements as specified in Chapter \ref{ch:design}.
It performs consistently, and faster than expected compared to a decent laptop, as shown in Chapter \ref{ch:results}.

% Content discovery
From the content discovery experiment, we learned that 9 devices discover new content within 4 seconds.
Therefore, with viral spreading, it seems realistic we can reach millions within minutes.
% Multichain performance
The Multichain experiment showed that creating blocks by the thousands is an IO bound process, rather than CPU bound, in the current implementation.
It is also the most easy to optimize by keeping the hash of the last block for connected peers in memory .
% Startup time
In the startup time experiment, the sample standard deviation is relatively small for all devices, which indicates that the startup time of Tribler is consistent.
% Content creation
As expected, creating torrents appears to scale linearly with the size of the content, and adding a torrent to a channel does not scale with content size.
% Api responsiveness
The API performs consistently, but slower on a smartphone than on a decent laptop.
The API also performs consistently slower, both on the laptop and the smartphone, depending on the amount of data to be returned for a request, as expected because of JSON serialization.
% Profilng
Profiling revealed that cryptographic tasks are a significant part of processing messages.
These tasks should be offloaded to a separate computational core to release the global interpreter lock.
That would enable Python code to run in parallel, which in turn would improve performance and responsiveness.
% Cpu utilization
The CPU utilization measurement showed this approach is likely to succeed, because the CPU utilization of Tribler and the video player VLC combined did not reach more 35\% while streaming HD video.
% Testing and coverage
All existing tests can be run on Android almost as well as on other platforms.

% System architecture


% Build toolchain


% Challenges


\section{Given the constraints and unique abilities of mobile devices, what functionality of Tribler can be added or enhanced?}
% Opportunities

% Bluetooth
For example, a feature was added that transfers Tribler to a nearby device, with NFC and Bluetooth enabled, without further requirements, like an Internet connection or a central app store.
% Nfc
By just holding to NFC equipped smartphones back to back, the transfer of the application started automatically.
Also, you can add a channel to your favorites in the same manner.
Using your real life social network, you can build your own on-line trust network this way.
% WiFi Direct
And thanks to built-in capability of Android you can setup an ad hoc WiFi network to avoid any infrastructure completely.
In the context of privacy and censorship, this off-grid functionality means that if content is detected by the censor it may have already crossed, or will cross, the freedom border thanks to the properties of viral spreading.


\chapter{Conclusions and future work}\label{ch:conclusions}
After the performance analysis of our implementation in the previous chapter, we now come to the conclusions with regard to the research questions.
Finally, in Section \ref{sec:future_work} we will describe possible avenues for future research based on this work.


\section{How feasible is it to run all Tribler functionality on mobile devices?} % to defeat or mitigate large scale monitoring and censorship?
We proposed an implementation on Android.
% System architecture


% Build toolchain


All design objectives are gehaald.
A new user group, that does not own desktop computers, can gain access to Tribler's privacy enhancing functionality.




% Content discovery
What can be concluded from this experiment is that on the same local network the first device discovers the new content in less than 2 seconds.
From the 10th device onward the dissemination slows down.
% Multichain performance
However this is an indication that creating blocks by the thousands is an IO bound process, rather than CPU bound.
% Startup time
The sample standard deviation is relatively small for all devices, which indicates that the startup time of Tribler is consistent.
% Content creation
Creating torrents appears to scale linearly with the size of the content, but adding a torrent to a channel does not scale with content size.
% Api responsiveness
The API performs consistently, but slower on a smart-phone than on a laptop.
The API also performs consistently slower, both on the laptop and the smart-phone, depending on the amount of data to be returned for a request, as expected because of JSON serialization.
% Profilng
The profiling revealed that crypto tasks are a significant part of processing messages.
These tasks should be offloaded to a separate computational core to release the global interpreter lock.
That would enable Python code to run in parallel, which in turn would improve performance and responsiveness.
% Cpu utilization

% Testing and coverage




% Challenges



The feasibility in terms of functionality, performance and scalability were measured.
Several of these measurements give raise to some minor considerations.
% Profiler
The profiling revealed that crypto tasks are a significant part of processing messages.
These tasks should be offloaded to a separate computational core to release the global interpreter lock.
That would enable Python code to run in parallel, which in turn would improve performance and responsiveness.
% cpu
The CPU utilization measurement showed this approach is likely to succeed.
% Multichain
The Multichain measurement showed that database performance in that instance was stringent in the current implementation.
It is also the most easy to resolve by keeping the hash of the last block for connected peers in memory .
% Tests
All existing tests can be run on Android just as well as on other platforms.
This means, in combination with the modularity of the new design and re-use of the Tribler core, that maintainability and testability are very fit.
% Startup
In terms of user experience the measurement of startup time showed to be very consistent and reasonable.
% Latency
The latency of the API however seemingly showed a curious pattern that cannot be explained right away.
It is only known to occur if hundreds of requests per minute are fired at the API, which is unexpected behavior from the user.

% op mobiel is API consistent and consistently slower than PC

\section{Given the constraints and unique abilities of mobile devices, what functionality of Tribler can be added or enhanced?}
% discovery is best snel



% Opportunities



Finally, to answer the secondary question: using the unique properties of mobile devices, what features can be added or enhanced?
% Bluetooth
We added the capability to transfer the app to another device that doesn't have it yet via Bluetooth.
% nfc
If NFC is enabled on both devices the app can be transfered by just holding the devices next to each other.
Also adding a channel to your favorites can be done this way.
Using your real life social network you can build your on-line trusted network this way.
% P2P Wifi
And thanks to built-in hardware capability you can setup an ad hoc WiFi network to avoid any infrastructure completely.


What this means in the context of privacy and censorship is that if content is detected by the censor it may have already crossed or will cross the freedom border thanks to the properties of viral spreading.


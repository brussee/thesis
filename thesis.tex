\documentclass[]{report}

\title{Autonomous operation of attack-resilient and self-organizing video-on-demand  platform for mobile devices}
\author{P.W.G. Brussee}

\begin{document}
\maketitle

\begin{abstract}
	Abstract
\end{abstract}

% related work = related scientific articles
% NOT engineering (tsap, at3, tribler-mobile (ppsp-twitter))
% https://scholar.google.nl/scholar?hl=en&q=video+on+demand+android&btnG=&as_sdt=1%2C5&as_sdtp=


\chapter{Introduction}
% 3 to 4 pages with screenshots
% related work in citations in between text

Video sharing is a big part of social media nowadays.
Huge video streaming providers like Youtube, Twitch, Periscope, etc. currently dominate the market.
None of them are server-less and do not  provide anonymity in any shape or form.
Social media platforms typically have an incentive to de-anonymize the user to serve targeted advertisement and sell the information about user interests with advertisers.


\section{Existing Mobile Streaming Software}
% NOT related work or problem description
From re-decentralized: top projects that match/compare.

Youtube screenshot, GNUnet


\section{Tribler}
Tribler introduces a server-less video-sharing platform with privacy enhancing technologies and giving a Youtube-like, social media experience at the same time.
The capability of hiding your identity is greatly advantageous to the user if his or her human rights are violated, like free speech.

The server-less technique of Tribler is resistant to Internet kill-switches that are typically deployed for the purpose of censorship.




\chapter{Problem description}
% FIRST and scientific, then figures and then filling in the holes

The business model of social media directly conflicts with user privacy. % FINAL

Social media on phones
You want to express freedom of expression with that all the time

Existing apps use central server design
Vulnarable for Internet kill switches and censorship

Offline viral spreading image (hacking lab)





\section{Scope}
No routing, no blutooth p2p, mesh networking: that is done by Serval. Out of scope of this work. Add as enhancement later.










\chapter{Architecture and Design}

Show complete technology stack


\section{Common Core}
10 years of Python code 
reusable multi-platform: Windows Mac Linux


Native Android app for Tribler with Java+xml GUI.
Using https://github.com/kivy/python-for-android to run Tribler python code as a native Android service.
The app communicates with the python service via the new rest api.

Picture of the technology stack and all components.


\chapter{Implementation}


we transfered the first generation bash script into a more involved python tool chain ecosystem
the p4a project accepted our first contribution the same day
our approach for this .. we focused on the low hanging fruit first to benefit from the learning effect
the first task consisted of porting the python bindings for 15 - 30+ commonly used libraries to Android [recipe example code block]

\subsection*{List of problems encountered}
cross-compile boost libs compileflags just PTHREADS
python bindings libtorrent soname conflict


sqlite3 builtin module (CPPFLAGS instead of CFLAGS, no patch setup.py after all, dispersy doesn't use aspw)


no shell acces from Popen() means only single binaries instead of normal commands

adding a native jni start environment shifts keyword arguments



\section{Python on Android}
Open source project Python-for-Android is capable of running python code.

\emph{My contributions:}
\begin{itemize}
	\item add service only bootstrap
	\item recipe (missing) for all dependencies
	\item bug fixes in build tool chain
\end{itemize}


\section{Rest API}

\emph{My contributions:}
\begin{itemize}
	\item create channel
	\item create torrent
	\item add torrent to channel
	\item async download and add torrent to channel from url and magnet links
	\item remote shutdown
\end{itemize}

\section{Common Core}

\emph{My contributions:}
\begin{itemize}
	\item correct paths for Android
	\item setup.py rewrite
	\item improve error handling in test runner
	\item various threading bug fixes
\end{itemize}


\section{Native Android App}
Gui for common core back-end and rest api in between.

\emph{My contributions:}
\begin{itemize}
	\item Native service wrapper for Python process
	\item GUI:
	\item create own channel
	\item record video
	\item create torrent file
	\item add torrent to own channel
	\item search torrents and channels
	\item voice search integration
	\item view channel contents
	\item download torrent
	\item launch video viewer
	\item mark and un-mark channel as favorite
\end{itemize}



\chapter{Performance Analysis}

\section{Typical System Load}
Graph showing cpu and memory consumption from fresh install to an hour idling. No tunnels.

Graph showing cpu and memory consumption with max. 10 tunnels after an hour idling.

Graph showing cpu and memory consumption during single download after an hour idling. No tunnels.

Graph showing cpu and memory consumption during streaming HD video after an hour idling. No tunnels.


\section{CProfiler Analysis}
Graph showing wall clock time spend on functions running 10 minutes during first half hour with max. 10 tunnels.


\section{Test Suite Performance}
Graph showing total run time and average run time per test.


\section{Multi-chain Scalability Experiment}
Graph showing scalability or lack thereof of the multi-chain record creation cost.


\chapter{Conclusions and Future Work}
app needs wifi access points


\end{document}          

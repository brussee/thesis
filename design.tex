\chapter{Design \& Architecture}

%OO book

knowledge of requirements
knowledge of design as created so far
knowledge of technology available
knowledge of software design principles and best practices
knowledge of what has worked well in the past


\section{Design principles}

Top down: high-level issues like software architecture and kind of database, down to details like format of data items and individual algorithms

Bottom up: decide reusable low-level utilities and then how these are put together to create high-level constructs

\subsection{Divide and conquer}

\subsection{Increase cohesion where possible}

functional
layer
communicational
sequential
procedural
temporal
utility

\subsection{Reduce coupling where possible}

content
common
control
stamp
data
routine call
type use
inclusion/import
external

\subsection{Keep the level of abstraction as high as possible}

\subsection{Increase re-usability where possible}

design for reuse

\subsection{Reuse existing designs and code where possible}

design with reuse

reuse of expertise
reuse of standard designs and algorithms
reuse of libraries of classes or procedures, of of powerful commands built into languages and operating systems
reuse of frameworks
reuse of complete applications

\subsection{Design for flexibility} %adaptability

reduce coupling and increase cohesion
create abstractions
no hard-coding
leave all options open, like exception handling by caller instead of callee
use reusable code and make code reusable, like hooks

\subsection{Anticipate obsolescence}

avoid early releases
avoid environment dependencies
avoid undocumented features or little used features, like libtorrent geo
avoid smaller companies/projects due to lack of long term support
use standard languages and technologies supported by multiple vendors

\subsection{Design for portability}

\subsection{Design for testability}

avoid static

\subsection{Design defensively}

check all inputs and preconditions
design by contract !!!
preconditions
postconditions
invariants
-> assertions
paper Analysing inter-application communication in Android
memory leak detection?


\section{Define objectives} %sub-questions of problem definition
%CLOSE GAP: connect all the puzzle piezes that is maintainable, sustainable, usable, ... and it all has to work on mobile
%!! refer back to problem definition
% good software product satisfies: ..... quality assurance/control %REF %to confront the product 

%requirements

%soort roos, wat is 'raak'

The key performance indicators (KPI) will be \emph{scalability}, \emph{resource usage} of cpu, memory and power, and \emph{usability} in terms of latency.

%SA book

\subsection{Accessibility}

\subsection{Availability and resilience}

\subsection{Development resource}

\subsection{Evolution}

\subsection{Internationalization}

\subsection{Location}

\subsection{Performance and scalability}
Just like the current server-centric design of social media
A fully distributed system needs to scale to potentially all smart phones in the world, just like the current architecture of social media with central server locations.

\subsubsection{Resource usage}

\subsection{Regulation}

\subsection{Security}

\subsection{Usability}





\section{Define alternatives} %design space %solution space
%low prio
> why existing technology is not sufficient to
> meet the described demands. The example proposed was the tor onion
> network in combination with XMPP or the orbot smartphone app. After
> much discussion the conclusion was that existing technologies, such as
> tor facilitate protected point-to-point communication. However,
> possible desired use cases focus more on current Twitter-like social
> media practices, best typified as a "global conversation".
> Furthermore, current social media revolves around video-rich,
> real-time interaction with groups, hashtag-based discovery and social
> networking. All of these aspects are not offered or are incompatible
> with current-generation of privacy enhancing technology

%just python+kivy OR java+xml+python+jni, etc...


%scope
Why no iOS?
No routing, no blutooth p2p, mesh networking: that is done by Serval. Out of scope of this work. Add as enhancement later.


\section{Architecture design}

better understanding
individual pieces worked on in isolation
prepare for extension
facilitate reuse and re-usability

logical breakdown into subsystems, package diagrams
dynamics of interaction among components at run time
data shared among subsystems
components existing at runtime and machines/devices on which they are located

Architectural patterns:

\subsection{Client-Server}
REST API

\subsection{Broker}
objects are actually remote

\subsection{Transaction processing}
atomicity

\subsection{Pipe-and-Filter}
Rx data flow pipeline

\subsection{Model-View-Controller}
model: JSON de-serialized objects
view: xml layout
controller: activity, fragment, adapter

\subsection{Service-Oriented}
app is collection of services that communicate with each other through well defined interfaces
http request behind the scenes
JSON REST API
videoserver

\subsection{Message-Oriented}
message-oriented middle-ware (MOM)
communicating apps do not have to be available at the same time
virtual channels, topics
publishers, broadcasters
subscribe to topic
android intents, intent filters, broadcasts, bundle, messages, handlers

\section{Class design}

\section{User interface design}

\section{Database design}

\section{Algorithm design}

\section{Protocol design}


\section{Requirements} % 

\subsection{Functional requirements} % Function design
%Product design involves determining the objectives of the system. Central questions in this process are “what is feasible?”, “what is required?” and “what functions need to be fulfilled by the system?” It is a strategic decision process and it will be termed the function design 
%Function design distinguishes between innovation and improvement. Improvement usually only concerns the reorganisation or reengineering of an existing system, which implies a rearrangement of existing functions or a different interpretation of these functions. Innovation concerns the extension, reduction or modification of functions due to the introduction of new technology, resources and/or organisation.

inputs, commands and conditions
outputs, conditions
store, reuse data, backup
computations
timing and synchronization

\subsection{Quality requirements}

response time
throughput
resource usage
reliability
availability
recover from failure
maintainability and enhancement
re-usability

\subsection{Platform requirements}

computing platform: min. system specs and features, api level
technology to be used: programming language, db

\subsection{Process requirements}

development process / methodology
cost
delivery date

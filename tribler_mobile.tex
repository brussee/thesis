\section{Towards Tribler on mobile devices}\label{ch:tribler_mobile}

In the previous sections we discussed the features and applications of Tribler.
So far Tribler only supports desktop and server versions of Linux, Mac and Windows.
The necessity of moving to mobile devices calls for Tribler functionality to be enabled to run on these resource limited devices.
All positive and negative properties that come from these features are transferred to mobile devices, which will add their own distinct properties to the mix.


% In terms of Functionality

\subsection{Opportunities}

What does mobile allow in addition to desktop?
% Portable
Mobile devices are inherently easy to move around and very portable.
% Ad hoc WiFi
In case of a breakdown in communication infrastructure the mobile devices with wireless radio transmitters can still connect ad hoc and moved within range if necessary.
% WiFi
Via WiFi a device can connect to the Tribler network via existing infrastructure or other peers via ad hoc WiFi.
% Bluetooth
Using NFC Tribler can start a Bluetooth connection and transfer the installation package peer-to-peer.
% nfc
Also via NFC Tribler can exchange channel ID's to subscribe to another Tribler channel peer-to-peer.

Bringing Tribler to mobile devices will give potentially millions of users access to these features, on the move.
Expanding the Tribler network with mobile devices could also benefit the research that can be performed on the live network.


\subsection{Challenges}

What does working on mobile phone mean?
The portability of mobile devices requires any network interface and power supply to be wireless.
% Battery
Mobile devices are typically equipped with batteries to operate without a power cord.
Considering the size and weight the capacity is limited.
Smart-phone batteries usually barely hold a charge that would sustain a day of heavy usage.
Tribler could potentially drain the battery much faster than normal.
% Crypto
Heavy encrypted network traffic not only demand constant radio transmissions, but also CPU processing.
In case of hidden seeding building circuits of 3 tunnels with layered encryption quadruples  the amount of cryptographic work.
% disk IO
Because Multichain punishes cheating like double spending by a permanent ban, it must never loose information and flush everything to permanent storage before continuing.
Mobile devices typically have flash memory with limited write-cycles compared to classic hard drives that are commonly found in desktop computers.


\chapter{Problem description}
% News & generic usage

The goal of Tribler is to become an information sharing platform / video-on-demand platform that protects the privacy of its users and is resilient to attacks while not relying on existing infrastructure.
The goal of this thesis project is to enable all Tribler functionality on mobile Android devices.


\section{Shift to social media}
% Consumption
News and media consumption is shifting from traditional outlets like TV broadcasts, newspapers and other physical media to digital media and mobile devices.
% Production
These digital mobile devices are also capable of recording multi-media content that can be directly shared from the device.
Increasingly news and media production is shifting from traditional outlets like news desks and investigative journalism to social media like Twitter, Snapchat and Facebook.
% Distribution
It is the same device that people use for consumption and producing content is capable of sharing news and multimedia.
Eye-witnesses are able to share multimedia content with others through social media.
News and information distribution is shifting from traditional media to social media.
% Editorial / Journalism / Processing
The users themselves are turning from con-sumers into pro-sumers.
No news desk or professional equipment is necessary to relay news directly from eye-witnesses to the masses anymore.
News and consumed media over social media is to a lesser and lesser degree subject to expert opinion, editorial and adversarial journalism. \cite{reuters_social_media}
% Social journalism
People with camera phones can reach 20 million people with multi-media in a very short timespan.
Social media has been a driving force in the Arab Spring.
A global dialog is possible through social media.
Large portions of the global dialog on social media is uncontrolled by traditional media or governments.


\subsection{Rise of the smart-phone}
% What advantages does a smart phone have compared to a laptop and desktop?
Given this context mobile devices are increasingly important and smart-phone in particular.
A smart-phone has the unique property of being a ubiquitous device that is highly mobile and extremely connectible.
Most smart-phones even have one or more cameras to produce multi-media content that can be shared immediately from the device.
Finally the entire world has a smart-phone.
Even or especially areas without traditional infrastructure they are ubiquitous.


\section{Communication during crisis}
% Single point of failure
With servers central to the design of typical communication networks they create a single point of failure, even in a decentralized set-up.
Several natural disasters have taken out the necessary infrastructure on numerous occasions for a prolonged period of time.
Especially in situations like these, people need to communicate and coordinate their efforts to restore safety.

In crisis situations, like natural disaster or unrest, the smart-phone becomes particularly important, exactly for the earlier mentioned properties.
Under certain conditions utilizing centralized infrastructure is undesired because of overloading the emergency communication network or physically impossible.

Social media has played a major role in recent calamities when people could mark themselves as safe, effectively broadcasting that information to all their family and friends on social media, instead of contacting them one by one or not at all due to congestion in the communication channels.


\section{Censorship}
% Single point of entry
Internet exchange (IX) infrastructures are among the central components in the inter-network architecture that are also vulnerable to monitoring, censorship and Internet kill-switches.
As such, not everyone has unrestricted access to the Internet due to censorship and surveillance.
In fact a significant part of today's Internet users is affected by these attempts to hide or distort reality. %ref
This interference directly affects the universal right to freedom of opinion and expression as stated in article 19 of the Universal Declaration of Human Rights (UDHR).

The Internet makes it easy to communicate freely on a global scale.
Connecting to it and crossing international borders on-line does not require approval of any governmental body. % wrong!
This freedom due to the absence of oversight and control allows anyone with the capability to monitor, filter, delay, or block Internet traffic at will.


\subsection{Invasion of privacy}
% Large scale monitoring

The lack of anonymity becomes a problem when the users privacy is being invaded.
Revealing personal information can be deduced from search queries for example, or associations on social platforms.
When this information can be used for targeted advertising it becomes very valuable, and creates an incentive for the parties that have access to this information to sell it to third parties.
In fact the business model of social media appears to be serving targeted advertisements to its users on behalf of third parties.
What's even worse is social media integrated into regular websites to de-anonymize and track the whereabouts of users even outside of the social media realm.
Whenever users lose control over their privacy it becomes a serious problem.

% Privacy
Pervasive monitoring of digital citizens by Internet providers on behalf of governments to enforce censorship laws raises severe privacy concerns.
Even the business model of social media companies directly conflicts with user privacy.
Targeted advertising requires the very information of high quality (accurate and current) users tend to share with their friends on-line.
When this information is shared with other parties outside of the specific social media website, possibly unknowingly to the user, it effectively becomes a privacy leak.
Subsequently users can be confronted with their information being misused in various ways beyond their control.
This lack of control over your own privacy can lead to arbitrary interference as defined in UDHR article 12. %ref, example, human rights watch, nelie kroes, etc.
Integration of social media on regular websites aggravates this problem.
Every page-view and click on social media enabled websites becomes traceable to an individual, directly benefiting the business model of targeted advertisements

% Internet censorship 2
The incentive to de-anonymize the user, not only causes a lack of privacy, but also a potential lack of freedom of expression, as it hands key information to the censor: who is expressing dissent and who is associated with this person on-line.
Cyber suppression has become a reality when you no longer can be associated with opinion-makers or foreign journalists on-line.


The sophistication of censorship techniques is pushed forward by the drive to stay ahead of attempts trying to circumvent it.
Increasingly though, Internet traffic is put under surveillance and obfuscation techniques are targeted by restrictions.


\subsection{Adversary model}
To ensure that no controlling party can exercise censorship we distribute authority over all users, creating an \emph{autonomous} system.
If all information is located in one or a few places, the parties in charge of that location will still have control over it, so we must distribute information over all users, creating a \emph{communication} system.
Then if all users want to use this system to share, order and appreciate each others information, in other words the essence of social media: social interaction, with everyone being able to interact in the same way, we need to  distribute functionality over all users, creating a \emph{cooperation} system.

% Solution
Fully distributed systems capture these characteristics.
Without any central component in the system it is no longer susceptible to censorship without everyone participating.

Decentralized can work in these situations.
Censorship and large scale monitoring is difficult in decentralized networks.
The Arab spring has shown that social media can work. \cite{Johan_2001}

Peer-to-peer communication technology is essential for a server-less distributed system.
Mobile devices typically do not require infrastructure to exchange information, like those equipped with Bluetooth or capable of ad hoc Wi-Fi.
Smart phones are ubiquitous everywhere in the world and used to access social media and retrieve information from the Internet.
Fortuitously these are also the type of mobile devices that can communicate peer-to-peer.



\section{Explore environment}
% learn about domain
% current state of technology
% what others before me have researched and accomplished

% Fragmented
Various initiatives have been started to deal with one or both of these problems. \cite{re_decentralize}
%ToDo: From re-decentralized: top projects that match/compare.

Figure \ref{fig:youbroketheinternet} shows a mapping of projects that are or have been working on that.
The fragmentation is clear from the figure and none of them provide a full solution.

%being first
%engineering perfection
%time to market
%popularity
%make a difference
%Tribler mobile vs periscope
%fully functional prototype, but poor user interface (unpolished)
%real world impact proven insufficient
%we want to change the world implicit


% Refereren aan eigen werk versplintering in oplossingen
There is no de-facto solution available on mobile platforms. \cite{literature_survey}
Previous research has shown that there is no solution that solves the problem entirely in a sustainable way.
Tribler is our attempt to solve this problem entirely in a sustainable way.
To see if it is feasible to apply it in a mobile context as described above, we need to port it.

% Previous Tribler-mobile app's
Previous attempts failed to deliver all functionality.
Maintainability issues with earlier designs were a large part of the reasons why.
We changed the architecture of Tribler for our approach.


% Contributions
By making Tribler available on mobile devices all research, with regard to the problems that Tribler tries to solve, now becomes possible on the mobile platform.

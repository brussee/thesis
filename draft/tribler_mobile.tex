\chapter{Towards Tribler on mobile devices}

In the last chapter we discussed the usefulness of Tribler functionality running on desktop and server machines.
To bring that to mobile devices will give access to the usefulness of Tribler for people on the move.
Expanding the Tribler network would also improve the usefulness of experiments on the live network for this large scale system.

Properties (positive and negative) that come from these features are transferred to mobile devices, which will add their own distinct properties to the mix.


How to create a \emph{self-organising} \emph{video-on-demand} platform that is \emph{attack-resilient} and can \emph{operate autonomously} on a \emph{mobile device}?

Mobile device in the sense that it is low-powered and portable including the network interface and power supply.


% In terms of Functionality

\section{Opportunities}

What does mobile allow in addition to desktop?

Mobile devices are inherently easy to move around and very portable.
In case of a breakdown in communication infrastructure the mobile devices with wireless radio transmitters can still connect ad hoc and moved within range if necessary.

Via WiFi a device can connect to the Tribler network via existing infrastructure or other peers via ad-hoc WiFi.
Using NFC Tribler can start a Bluetooth connection and transfer the installation package peer-to-peer.
Also via NFC Tribler can exchange channel ID's to subscribe to another Tribler channel peer-to-peer.
In the future the same method can be used to exchange bootstrap peers for the Tribler network.
The peer-to-peer Bluetooth file transfer could be made much faster if the WiFi Direct would be used.


\section{Challenges}

What does working on mobile phone mean?

Mobile-devices are typically equipped with batteries to operate without a power cord.
Considering the size and weight the capacity is limited.
Smart-phone batteries usually barely hold a charge that would sustain a day of heavy usage.
Tribler could potentially drain the battery fast.
Heavy encrypted network traffic not only demand constant radio transmissions, but also CPU processing.


3 tunnels = 3x heavy
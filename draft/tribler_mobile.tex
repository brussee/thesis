\chapter{Towards Tribler on mobile devices}

In the last chapter we discussed the usefulness of Tribler functionality running on desktop and server machines.
To bring that to mobile devices will give access to the usefulness of Tribler for people on the move.
Expanding the Tribler network would also improve the usefulness of experiments on the live network for this large scale system.

Properties (positive and negative) that come from these features are transferred to mobile devices, which will add their own distinct properties to the mix.


How to create a \emph{self-organising} \emph{video-on-demand} platform that is \emph{attack-resilient} and can \emph{operate autonomously} on a \emph{mobile device}?

Mobile device in the sense that it is low-powered and portable including the network interface and power supply.


% In terms of Functionality

\section{Opportunities}

What does mobile allow in addition to desktop?
% Portable
Mobile devices are inherently easy to move around and very portable.
% Ad hoc WiFi
In case of a breakdown in communication infrastructure the mobile devices with wireless radio transmitters can still connect ad hoc and moved within range if necessary.
% WiFi
Via WiFi a device can connect to the Tribler network via existing infrastructure or other peers via ad-hoc WiFi.
% Bluetooth
Using NFC Tribler can start a Bluetooth connection and transfer the installation package peer-to-peer.
% nfc
Also via NFC Tribler can exchange channel ID's to subscribe to another Tribler channel peer-to-peer.
The same method can be used to exchange bootstrap peers for the Tribler network.
% WiFi transfer
The peer-to-peer WiFi Direct file transfer would be much faster than Bluetooth.


\section{Challenges}

What does working on mobile phone mean?
% Battery
Mobile devices are typically equipped with batteries to operate without a power cord.
Considering the size and weight the capacity is limited.
Smart-phone batteries usually barely hold a charge that would sustain a day of heavy usage.
Tribler could potentially drain the battery much faster than normal.
% Crypto
Heavy encrypted network traffic not only demand constant radio transmissions, but also CPU processing.
In case of hidden seeding building circuits of 3 tunnels with layered encryption quadruples  the amount of crypto work.
% disk IO
Because Multichain punishes cheating like double spending by a permanent ban, it must never loose information and flush everything to permanent storage before continuing.
Mobile devices typically have flash memory with limited write-cycles compared to classic hard drives that are commonly found in desktop computers.


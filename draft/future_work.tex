\chapter{Conclusions}
After validating and verifying our design in the previous chapter, we now come to the conclusions with regards to the main research question.

The main research question was: how feasible is it to run all Tribler functionality on mobile devices? % to defeat or mitigate large scale monitoring and censorship?

The feasibility in terms of functionality, performance and scalability were measured.
Several of these measurements give raise to some minor considerations.
% Profiler
The profiling revealed that crypto tasks are a significant part of processing messages.
These tasks should be offloaded to a separate computational core to release the global interpreter lock.
That would enable Python code to run in parallel, which in turn would improve performance and responsiveness.
% cpu
The CPU utilization measurement showed this approach is likely to succeed.
% Multichain
The Multichain measurement showed that database performance in that instance was stringent in the current implementation.
It is also the most easy to resolve by keeping the hash of the last block for connected peers in memory .
% Tests
All existing tests can be run on Android just as well as on other platforms.
This means, in combination with the modularity of the new design and re-use of the Tribler core, that maintainability and testability are very fit.
% Startup
In terms of user experience the measurement of startup time showed to be very consistent and reasonable.
% Latency
The latency of the API however seemingly showed a curious pattern that cannot be explained right away.
It is only known to occur if hundreds of requests per minute are fired at the API, which is unexpected behavior from the user.

Finally, to answer the secondary question: using the unique properties of mobile devices, what features can be added or enhanced?
% Bluetooth
We added the capability to transfer the app to another device that doesn't have it yet via Bluetooth.
% nfc
If NFC is enabled on both devices the app can be transfered by just holding the devices next to each other.
Also adding a channel to your favorites can be done this way.
Using your real life social network you can build your on-line trusted network this way.
% P2P Wifi
And thanks to built-in hardware capability you can setup an ad hoc WiFi network to avoid any infrastructure completely.



Future Work

shared keychain, credit minging laptop, server, mobile..


NFC is used to share a channel id and to automatically setup a Bluetooth file transfer.
In the future the same method can be used to exchange bootstrap peers for the Tribler network.
The peer-to-peer Bluetooth file transfer could be made much faster if the WiFi Direct would be used.


Thumbnails retrieval and distribution now that video is so important on mobile, could be added on top of groundwork by me.


credit mining on device A, usage on device B


comparative analysis
WiFi direct
cutting long running methods in smaller pieces, so multi-threading is smoother (only discovered now, earlier: why is gui hanging?)
release GIL in native C code, mainly for heavy crypto tasks
streaming API for big responses (dumping all known channels from db)

knippen voor twisted is gewenst, nu pas zichtbaar weakness in Tribler, nuttig om te verbeteren. (conclusion)


% Research Limitations
% Project scope
% broad / narrow scope, scope of other system()s)
% sub-problems
% ultimate high-level goal
Software development / technical aspect only
Not policy making, organisational perspective, decision making, normative, ethical,
Time limit of 9 months


refereren aan eigen werk in verbergen app en mensen beschermen op bepaalde manier
combine with selfcompileapp
%mooie afsluiting, boog.
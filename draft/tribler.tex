Tribler features


using UDP rather than TCP.
This privacy feature requires a lot more bandwidth of the network than without anonymity: a ratio of 13 GB for every anonymous 1 GB of data.
Since bandwidth is limited and transitory it can be beneficial to exchange unused bandwidth for a promise of bandwidth in the future, or another reward.
The research group behind Tribler is currently building a fully decentralized accounting system and open exchange market using block-chain technology with the purpose of building trust on-line and creating the Internet of Money.
% not only benefit the users thanks to an increase of available bandwidth, but more importantly  MAYBE more users, more bandwidth necessary. % Credit mining on device A to be consumed on device B.
Expanding the Tribler network would improve the usefulness of experiments on the live network for this large scale system.


\subsection{Tribler} %desktop version
% tribler and dispersy intro, what is it (read thesis independently)
%EXPLAIN what has been done to fix attack-reselience, VOD, self-organising, etc...

Tribler introduces a server-less video-sharing platform with privacy enhancing technologies and giving a Youtube-like, social media experience at the same time.
The capability of hiding your identity is greatly advantageous to the user if his or her human rights are violated, like free speech.

The server-less technique of Tribler is resistant to Internet kill-switches that are typically deployed for the purpose of censorship.

%what is the gap?
%GAP: no connection between puzzle pieces, that actually works on mobile


What is VOD?
What are the existing modules we re-use?


\section{Thesis definition}
The main question thus becomes:
How to create a \emph{self-organising} \emph{video-on-demand} platform that is \emph{attack-resilient} and can \emph{operate autonomously} on a \emph{mobile device}?

Self-organising in the sense that the platform coordinates the exchange of videos and meta-data fully automatically.

Video-on-demand in the sense that users can simply click and play videos in a streaming fashion, so without waiting for the entire video to be present on the device.

Attack-resilient in the sense that:
First: censorship does not have an effect if the majority of users does not cooperate with the censor.
Second: the privacy of users remains protected while they actively participate on the platform
Third: no network infrastructure required for viral spreading of the entire video platform.

Autonomous operation in the sense that users do not have to manage any files or configuration manually at all to be active on the platform.

Mobile device in the sense that it is low-powered and portable including the network interface and power supply.

These properties will ensure social media with resilience against Internet kill switches, natural disasters and censorship.


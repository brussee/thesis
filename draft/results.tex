
To verify if running all Tribler functionality on mobile devices is feasible we take a look at relevant performance characteristics and take several measurements.
For scale a laptop and old computer are included in some measurements.
We focus on the design aspects and evaluate per feature / key performance indicator.

% Rationale
% Metrics
% Expected / desired results
% Setup
% Results
% Conclusions

\section{API response time}
% Rationale
The user expects operations to take a consistent and reasonable amount of time.
By design all functionality is going through the API.
% Metrics
To test if the API is adding a significant amount of delay we measure the latency.
% XOR: latency vs response time
The response time is measured to verify if the API is responding consistently within a reasonable amount of time.
This includes the round trip latency.
Latency is the amount of time between sending a request and receiving it at the destination.
It excludes the time needed to process the request and latency on the way back.
Round trip latency includes the latency of the response.
The response time is the sum of the round trip latency and the processing time at the destination.
% Expected / desired results
We want to see that response times are bounded, consistent and generally low.

Previous research has shown the following results on 
eerder gedaan bij Laurens, daar dit en dat gezien
hetzelfde gedaan op Andorid / mobile

% Setup
To measure this we request the discovered channels from the API 1000 times and measure the response time. As done before by Laurens REF
A Nexus 6 smart-phone with Android 6.0.1 Cyanogen mod was used in this benchmark.

% Results
The following figure shows the latency for every request.

1 graph
1 device

% Conclusions
The results shows large spikes and an unexpected pattern.
It turns out this approach still suffers from performance issues due to imperfect multi-threaded coding of Tribler.
Looks like more things are at play here.
CPU freq?
Create and show laurens' histogram from his data. difference!


\section{Testing and coverage}
% Rationale
The design choice of reusing all Tribler core source code means we need to verify its correctness.
To make sure all code on Android works the same as on other supported platforms we need to test all code.
Tribler has some unit tests and integration tests that cover a large portion of the code, but not all.
% Metrics
% Expected / desired results
% Setup
The nosetests module together with the coverage module are used to evaluate this.
A nexus 

% Results
The followinng table shows the results of two times.

1 table
1 device

1.
testsuite name="nosetests" tests="711" errors="14" failures="13" skip="30"
coverage branch-rate="0" line-rate="0.7241" timestamp="1468681674588" version="4.1"
2.
testsuite name="nosetests" tests="749" errors="12" failures="15" skip="3"
coverage branch-rate="0" line-rate="0.7861" timestamp="1478016138747" version="4.1"

% Conclusions







Profiler
% Rationale
% Metrics
% Expected / desired results
% Setup

1 graph
1 device
wall clock time
10 minutes run

% Results
Results as expected: crypto lets other things wait, significant part of wall-clock-time.

% Conclusions


Startup time
% Rationale
% Metrics
% Expected / desired results
% Setup
% Results
% Conclusions
1 graph
5 devices


Multichain
% Rationale
% Metrics
% Expected / desired results
% Setup
% Results
% Conclusions
4 graphs
5 devices

The performance of Multichain:
% block creation graph
3 plots van onderling vergelijkbare resltaten
waarom moet dit experiment er zijn?
Om mobiele devices met Tribler volwaardige nodes in het netwerk te maken moeten ze multichain aankunnen ook na 10 jaar draaien.
Load test, simulatie 10 jaar. (174 days!)

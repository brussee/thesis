\chapter{Verification and validation}
% Validation: Are we building the right system?
% Verification: Are we building the system right?
To verify if running all Tribler functionality on mobile devices is feasible we take a look at relevant performance characteristics and take several measurements.
For scale a laptop and old computer are included in some measurements.
We focus on the design aspects and evaluate per feature / key performance indicator.

% Rationale
% Metrics
% Expected / desired results
% Setup
% Results
% Conclusions

\section{API latency}
% Rationale
The user expects operations to take a consistent and reasonable amount of time.
By design all functionality is going through the API.
% Metrics
We use Apache JMeter to verify if the API is responding consistently within a reasonable amount of time by measuring the latency.
JMeter measures the latency from just before sending the request to just after the first response has been received. \cite{jmeter_glossary}
Thus the measurement includes the time needed to assemble the request as well as processing it and returning a response with latency on the way back.
It excludes the transfer time of the complete response and subsequent processing and rendering time.
Any client can process and render a response differently, possibly in a streaming fashion.
By leaving out that element this metric is measuring the operation time via the API.
% Expected / desired results
We want to see that the latency is bounded, consistent and generally low.
Previous research has shown the following results on a computer:
%FIG api_laurens
% Setup
We did the same benchmark with a slightly different setup.
A Nexus 6 smart-phone with Android 6.0.1 Cyanogen mod was connected via USB to a laptop running JMeter.
The API tcp-port was forwarded with ADB.
With JMeter we request the discovered channels from the API a 1000 times and measure the latency.
JMeter is also capable of setting a desired number of requests per minute, provided the device can handle it.
We set this parameter to 300 requests per minute, equal to the other benchmark.
% Results
Our database to be queried is not equal to one used in figure \ref{fig:api_laurens}.
The following figure shows the latency for every request.
%FIG api_benchmark
The plot shows large spikes and seemingly a curious and unexpected pattern.
Table \ref{table:api_benchmark} shows the large difference in statistics.
\begin{table}[h]
  \begin{tabular}{l | *{8}{r}}
  	 & N & \overline{x} (ms) & Min. (ms) & Max. (ms) & σ (ms) & Throughput & KB/second & Avg. Bytes \\ \hline
	Nexus 6 & 1000 & 1803 & 1292 & 3292 & 301.25 & 33.3 req/minute & 347.44 & 641709.0 \\ \hline
	PC         & 1000 & 5        & 3      & 107   & 4.42     & 140.8 req/second & 2548.00 & 18525.0 \\ \hline
  \end{tabular}
  \caption[Total response time statistics]{These numbers are based on elapsed time, not latency, from the entire benchmark.}
  \label{table:api_benchmark}
\end{table}
The set rate of 300 requests per minute is clearly not reached by the smart-phone in our benchmark by a factor of 9.
The difference in amount of response data turned out to be much bigger than expected with a factor of 35.
% Conclusions
Considering the difference in amount of response data and the bandwidth of USB versus internal memory it is an unfair comparison.
However, based on our result we may conclude that the API does not respond consistently and appear to behave differently than the other benchmark.
The device not being able to process 5 requests per second possibly reveals information about the connection bandwidth and the multi-threaded processing capability.
We assume that the 480MB/s theoretical bandwidth of USB2.0 is not a bottleneck considering the 2,5 MB/s result of the PC.
Being a mobile device also other aspects my be at play here, like CPU frequency scaling.
However this was turned off by acquiring a wake lock from the Android OS.
It appears then that our design approach still suffers from performance issues due to imperfect multi-threaded performance.
Because if the CPU is simply not powerful enough we expect to see a linear pattern instead of what we actually see in \ref{fig:api_benchmark}.
This phenomenon is not of great importance though, because users are not expected to fire hundreds of requests per minute to the API.
Further investigation should figure out if this phenomenon is also observed with lower amounts of requests per minute.


\section{Testing and coverage}
% Rationale
The design choice of reusing all Tribler core source code means we need to verify its correctness.
To make sure all code on Android works the same as on other supported platforms we need to test all code.
Tribler has some unit tests and integration tests that cover a large portion of the code, but not all.
% Metrics
The ratio of tested lines of code with respect to the total number of lines of code is the coverage line-rate.
% Expected / desired results
We expect to see a line-rate value close to 1, but not 1 since we know the tests do not cover everything.
% Setup
All tests were run two times on the same device with 11 weeks of development in between.
The nose module was used for running the tests together with the coverage module for gathering coverage data.
The same Nexus 6 smart-phone with Android 6.0.1 Cyanogen mod was used in both runs.
% Results
The following table shows the results of both executions.
\begin{table}[h]
	\begin{tabular}{l | *{5}{r}}
		Run & Tests & Errors & Failures & Skipped & Line-rate \\ \hline
		1     & 711   & 14       & 13          & 30          & 0.7241 \\ \hline
		2     & 749   & 12       & 15          & 3            & 0.7861 \\ \hline
	\end{tabular}
	\caption[Total response time statistics]{These numbers are based on elapsed time, not latency, from the entire benchmark.}
	\label{table:testing_coverage}
\end{table}
The number of tests has increased as well as the coverage line-rate while the number of errors and skipped tests have decreased.
A failure means an assertion was not met and an error represents an exception while running a test.
% Conclusions
Therefore seeing that the number of failures increased is not that bad since the number of errors decreased.
If the line-rate is 1 you still need the the branch-rate to be 1 as well before you can be confident the code will work as expected.
The branch-rate is the number of code paths tested with respect to the total number of code paths possible.
Unfortunately this metric is not part of the current test plan.
Nevertheless the metrics show improvement overall.


\section{Profiling}
% Rationale
Because of the challenges put forward in chapter \ref{ch:tribler_mobile} we investigate if time is spent disproportionately on some function.
% Expected / desired results
We expect that the limited resources of a mobile device may impact particular features more than others.
If hardware acceleration is not present the less powerful CPU may struggle with encryption tasks.
% Metrics
Instead of CPU time, time actually spent processing by the CPU, we measure wall-clock time.
This way we measure the amount of time a user would have to wait for a certain function to be executed.
% Setup
With the cProfile Python module and the visualisation tool SnakeViz we can see if any function takes a disproportionate amount of time.
A Nexus 6 smart-phone with Android 6.0.1 Cyanogen mod was used for profiling Tribler.
The profiler was running for 10 minutes with Tribler during normal operation and without any user input.
% Results
%FIG profile_1468515157-2

\begin{table}[h]
	\begin{tabular}{*{3}{r} | l}
		\hline
		# Calls & Total time & Time per call & Function \\\hline\hline
		\csvreader[separator=tab,late after line=\\\hline]
		{Profiler/native_functions.csv}
		{ncalls=\ncalls,tottime=\tottime,percall=\percall,function=\function}
		{\ncalls & \tottime & \function}
	\end{tabular}
	\caption[Profiling details]{Native function calls.}
	\label{table:profiling_details}
\end{table}
Figure \ref{fig:profile} shows that 27\% of the time is spent on verifying cryptographic signatures.
The bright pink represents the update function, which signifies various business logic upon receiving a torrent.
Also notable is the same stack of functions within and outside of a community on top of the wrapper.
This can be explained by the fact that torrents can be discovered outside of a community too.
% Conclusions
The significant chunk of time that the crypto takes is as expected.
Since this task is actually delegated to the C library M2Crypto it should be possible to release the GIL of the Python interpreter so other Python code that does not depend on it can be executed.
The main alternative provided in the standard library for CPU bound applications is the multiprocessing module, which works well for workloads that consist of relatively small numbers of long running computational tasks, but results in excessive message passing overhead if the duration of individual operations is short \cite{http://python-notes.curiousefficiency.org/en/latest/python3/multicore_python.html}.
As seen from the time per call for __m2crypto.ecdsa_verify in table \ref{table:profiling_details} the multiprocessing module would likely cause too much overhead.


\section{CPU utilization}
% Rationale
Python-for-Android supplies a CPython interpreter out of the box.
CPython is optimized for single thread performance and compatibility with C extension modules.
It is limited by a global interpreter lock (GIL) in multi-threaded use cases with shared memory.
Tribler uses C extension modules for crypto tasks, which are CPU intensive.
Tribler also uses the event-driven networking engine Twisted, which is written in Python.
The core of the event loop within Twisted is the reactor, which runs on a single thread.
The reactor provides a threading interface to offload long running tasks, such as IO or CPU intensive tasks to a thread pool.
The GIL prohibits more than one thread to execute Python bytecode at a time.
This negates all performance gains in terms of parallelism afforded by multi-core CPUs, making Python threads unusable for delegating CPU bound tasks to multiple cores.
As shown in the precious section the crypto function took a considerable amount of time to compute.
To see if the releasing the GIL as put forward as a solution is feasible we measure if the CPU has more capacity than is being utilized right now.
% Metrics
During normal operation equivalent to the profiler measurement, we take a snapshot of the CPU utilization.
% Expected / desired results
If there is any performance to gain by releasing the GIL the CPU must be under-utilized right now.
% Setup
The same Nexus 6 smart-phone with Android 6.0.1 Cyanogen mod was used for this measurement.
% Results
%FIG monitor
The results show that indeed not al 4 cores of the CPU are utilized by a large margin.
%FIG cpu_usage_2 (??)
Even when performing intensive crypto work of the Multichain experiment while running all tests the CPU 1 core is mostly idling.
% Conclusions
This meas that releasing the GIL during heavy crypto work could result in a significant performance gain.


\section{Startup time}
% Rationale
One of the design objectives of separating the front-end from the back-end was responsiveness.
If an user decides to use Tribler we want to put no obstacle in the way.
Therefore the service starts up in the background, separately from the GUI.
However before any task can be executed by the service it needs to be actually started.
% Metrics
To measure the time it takes Tribler to fully start we measure the time from starting the app up to the moment the Tribler-started-event is fired.
This events is sent over the API event-stream and signifies that Tribler is fully started and ready to accept all incoming requests.
% Expected / desired results
We expect consistent loading times because right after starting we shut Tribler down again.
% Setup
To get a good idea of how the user experience may differ we measure the startup time 10 times on 5 different devices.
% Results
%FIG startup_time
Table \ref{table:startup_time} shows the statistics per device.
\begin{table}[h]
	\begin{tabular}{l | *{5}{r}}
		Device & N & \overline{x} (s) & Min. (s) & Max. (s) & s (s) \\ \hline
		Nexus 10        & 10 & \\ \hline
		Nexus 6          & 10 & 4.319 & 4.124 & 4.670 & 0.179 \\ \hline
		Nexus 5          & 10 & 3.353 & 3.273 & 3.459 & 0.081 \\ \hline
		Galaxy Nexus & 10 & 7.086 & 6.161 & 7.772 & 0.454 \\ \hline
		S3                   & 10 & 31.935 & 30.616 & 33.940 & 1.116 \\ \hline
	\end{tabular}
	\caption[Total response time statistics]{These numbers are based on elapsed time, not latency, from the entire benchmark.}
	\label{table:starup_time}
\end{table}
The results show a very small sample standard deviation and a very low startup time.
The S3 is performing worse than may be expected judging from the results of the other devices.
% Conclusions
The reason for that may be that this phone was not wiped and given a fresh install of Android.
That could mean that other applications installed on a device could significantly impact the performance of Tribler.
The sample standard deviation is relatively small though, which contradicts this hypothesis.
This should be investigated further, including if anything can be done on the part of Tribler.


\section{Multichain performance}
% Rationale
Multichain is the new accounting system of Tribler.
This feature is central to the concept of trust in the Tribler network and is very important for the future as other functionality will be built upon it.
If mobile devices are to become full-fletched nodes on the network they must support this feature.
With Multichain any peer registers the bandwidth it exchanges with other peers.
It aggregates these exchanges in blocks and signs them like a receipt and sends that to the other party to sign as well.
These blocks are linked in a block-chain to foil attempts of cheating the system.
% Metrics
The creation and signing of these blocks is measured to determine if it scales well.
Multichain signs a block every 10 minutes, meaning our experiment of generating 10.000 blocks represent 69 days of continuous effort.
% Expected / desired results
The database containing these blocks will grow over time, but should not slow down too much because of it.
% Setup
Measurements were taken on six different devices on multiple moments during development.
% Results
The following figures show the performance graphs of every measurement.
%FIG multichain_1
%FIG multichain_2
%FIG multichain_3
%FIG multichain_25
Each figure contains mutually comparable results.
Clearly visible from all graphs is that Multichain does not scale linearly on any device.
They also show that mobile devices are at least a factor 2 slower than an old PC or laptop and scale worse.
The Nexus 5 being faster than the more powerful Nexus 6 at some points may be deemed odd.
Running the experiment twice causes a noticeable slowdown on all devices as shown in figure \ref{fig:multichain_3}.
% Conclusions
This could be explained by the fact that in between those runs the database was not wiped, but a different id is used for the communicating peers.
Due to the nature of block-chain every new block needs to contain the hash value of the previous block.
If a database lookup is needed for this and the database is growing, that can explain the non-linear course of the graph.
This can be easily resolved by keeping the last hash value for currently connected peers in memory.
The slower Nexus 6 compared to the Nexus 5 can be explained by the fact that the SQLite database is stored on flash memory, which may differ in performance between both devices.
This oddity will most likely disappear if the relevant database records are kept in memory.
However this is an indication that creating blocks by the thousands is an IO bound process, rather than CPU bound.
Finally, if mobile devices are to be full-fletched nodes on the Tribler network, they should not slow down significantly more than an old PC, besides being slower in the first place.
Hardware acceleration could close this gap without sacrificing battery life too much.
Because mobile devices are a bit behind on the technology curve with respect to desktop computers it is probable that the gap becomes smaller over the coming years.
The capacity to store enough Multichain blocks to audit past exchanges should also be on par.
If not, other more powerful nodes could be queried to supply the necessary history about a peer, that requests your bandwidth, to verify if that peer is trustworthy.



To verify if running all Tribler functionality on mobile devices is feasible we take a look at relevant performance characteristics and take several measurements.
For scale a laptop and old computer are included in some measurements.
We focus on the design aspects and evaluate per feature / key performance indicator.

% Rationale
% Metrics
% Expected / desired results
% Setup
% Results
% Conclusions

\section{API latency}
% Rationale
The user expects operations to take a consistent and reasonable amount of time.
By design all functionality is going through the API.
% Metrics
We use Apache JMeter to verify if the API is responding consistently within a reasonable amount of time by measuring the latency.
JMeter measures the latency from just before sending the request to just after the first response has been received. \cite{jmeter_glossary}
Thus the measurement includes the time needed to assemble the request as well as processing it and returning a response with latency on the way back.
It excludes the transfer time of the complete response and subsequent processing and rendering time.
Any client can process and render a response differently, possibly in a streaming fashion.
By leaving out that element this metric is measuring the operation time via the API.
% Expected / desired results
We want to see that the latency is bounded, consistent and generally low.
Previous research has shown the following results on a computer:
%FIG api_laurens
% Setup
We did the same benchmark with a slightly different setup.
A Nexus 6 smart-phone with Android 6.0.1 Cyanogen mod was connected via USB to a laptop running JMeter.
The API tcp-port was forwarded with ADB.
With JMeter we request the discovered channels from the API a 1000 times and measure the latency.
JMeter is also capable of setting a desired amount of requests per minute, provided the device can handle it.
We set this parameter to 300 requests per minute, equal to the other benchmark.
% Results
Our database to be queried is not equal to one used in figure \ref{fig:api_laurens}.
The following figure shows the latency for every request.
%FIG api_benchmark
The plot shows large spikes and seemingly a curious and unexpected pattern.
Table \ref{table:api_benchmark} shows the large difference in statistics.
\begin{table}[h]
  \begin{tabular}{l | *{8}{c}}
  	 & # Samples & Average (ms) & Min. (ms) & Max. (ms) & σ (ms) & Throughput & KB/second & Avg. Bytes \\ \hline
	Nexus 6 & 1000 & 1803 & 1292 & 3292 & 301.25 & 33.3 req/minute & 347.44 & 641709.0 \\ \hline
	PC         & 1000 & 5        & 3      & 107   & 4.42     & 140.8 req/second & 2548.00 & 18525.0 \\ \hline
  \end{tabular}
  \caption[Total response time statistics]{These numbers are based on elapsed time, not latency, from the entire benchmark.}
  \label{table:api_benchmark}
\end{table}
The set rate of 300 requests per minute is clearly not reached by the smart-phone in our benchmark by a factor of 9.
The difference in amount of response data turned out to be much bigger than expected with a factor of 35.
% Conclusions
Considering the difference in amount of response data and the bandwidth of USB versus internal memory it is an unfair comparison.
However, based on our result we may conclude that the API does not respond consistently and appear to behave differently than the other benchmark.
The device not being able to process 5 requests per second possibly reveals information about the connection bandwidth and the multi-threaded processing capability.
We assume that the 480MB/s theoretical bandwidth of USB2.0 is not a bottleneck considering the 2,5 MB/s result of the PC.
Being a mobile device also other aspects my be at play here, like CPU frequency scaling.
However this was turned off by acquiring a wake lock from the Android OS.
It appears then that our design approach still suffers from performance issues due to imperfect multi-threaded performance.
Because if the CPU is simply not powerful enough we expect to see a linear pattern instead of what we actually see in \ref{fig:api_benchmark}.


\section{Testing and coverage}
% Rationale
The design choice of reusing all Tribler core source code means we need to verify its correctness.
To make sure all code on Android works the same as on other supported platforms we need to test all code.
Tribler has some unit tests and integration tests that cover a large portion of the code, but not all.
% Metrics
% Expected / desired results
% Setup
The nosetests module together with the coverage module are used to evaluate this.
A nexus 

% Results
The followinng table shows the results of two times.

1 table
1 device

1.
testsuite name="nosetests" tests="711" errors="14" failures="13" skip="30"
coverage branch-rate="0" line-rate="0.7241" timestamp="1468681674588" version="4.1"
2.
testsuite name="nosetests" tests="749" errors="12" failures="15" skip="3"
coverage branch-rate="0" line-rate="0.7861" timestamp="1478016138747" version="4.1"

% Conclusions







Profiler
% Rationale
% Metrics
% Expected / desired results
% Setup

1 graph
1 device
wall clock time
10 minutes run

% Results
Results as expected: crypto lets other things wait, significant part of wall-clock-time.

% Conclusions


Startup time
% Rationale
% Metrics
% Expected / desired results
% Setup
% Results
% Conclusions
1 graph
5 devices


Multichain
% Rationale
% Metrics
% Expected / desired results
% Setup
% Results
% Conclusions
4 graphs
5 devices

without clearing the database, running the experiment a second time shows a significant delay on all devices.

The performance of Multichain:
% block creation graph
3 plots van onderling vergelijkbare resltaten
waarom moet dit experiment er zijn?
Om mobiele devices met Tribler volwaardige nodes in het netwerk te maken moeten ze multichain aankunnen ook na 10 jaar draaien.
Load test, simulatie 10 jaar. (174 days!)

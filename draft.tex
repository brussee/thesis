The Internet is "broken" due to large scale monitoring and censorship.
Social media is important to people for distributing news and opinion.
Social media is largely consumed on smart-phones (and news on a smart-tv).
A big part of news and social media is video.

To solve the issues with large scale monitoring and censorship, particularly on social media a fully decentralized system is proposed.

Tribler is a fully decentralized video-on-demand platform.
It uses network overlays, called communities, to offer features like search and manage contributions to channels, for discover-ability of content
It offers privacy through multiple layered encrypted tunnels similar the TOR network, using UDP rather than TCP.
This privacy feature requires a lot more bandwidth of the network than without anonymity: a ratio of 13 GB for every anonymous 1 GB of data.
Since bandwidth is limited and transitory it can be beneficial to exchange unused bandwidth for a promise of bandwidth in the future, or another reward.
The research group behind Tribler is currently building a fully decentralized accounting system and open exchange market using block-chain technology with the purpose of building trust on-line and creating the Internet of Money.

Expanding the Tribler network would not only benefit the users thanks to an increase of available bandwidth, but more importantly improve the usefulness of experiments on the live network for this large scale system.

Currently Linux, Mac and Windows are supported by Tribler. 
Since Android dominates the smart-phone market it is the first candidate to support for mobile devices.


The research question is: can Tribler be used on mobile devices to defeat large scale monitoring and censorship?
Secondary question is: using the unique properties of smart-phones, what features can be added or enhanced?

First, what is required to run Tribler on Android?
Tribler is written in Python and uses various libraries written in C or C++.
To use these libraries on a mobile device they need to be compiled for the right embedded-application binary interface (EABI) including all nonstandard dependencies.
Since ARMv7 is currently supported by most Android phones it is the first to be supported by Tribler for now.


https://kos.gd/posts/5-ways-to-use-python-with-native-code/
Calling C code directly from Python is possible by using the Python ctypes module to load a native dynamic-link library (.so files on Android) or by using the Python/C API of CPython directly.
This API enables a library to define functions that are written in C to be callable from Python.
These Python bindings are the glue between pure Python and pure C code.
SWIG can generate the boiler plate code for this.
Libtorrent, one of Triblers' main components, uses Boost.Python to provide a standard C++ API on top of the Python/C API.
Kivy uses Cython; P4A itself?

The API is actually so powerful it even provides access to the internals of the interpreter to mess with the global interpreter lock (GIL) which could be released during native C calls to improve the multi-threading performance of Tribler crypto.

The Python-for-Android tool-chain provides a way (called recipes) to build these libraries and bindings with the required build tools.

P4A uses the Java native interface (JNI), between Java and C/C++ code, to launch the CPython interpreter from a thin Java Android application.
JNI enables to define functions that are written in C to be callable from Java.


Since API level 18 (Android 4.3 codenamed Jelly Bean MR2) loading native libraries dynamically from Python works properly.
Earlier versions of Android supported NDEF Push with NFC and Wi-Fi P2P since API level 14, and initiate large file transfers over Bluetooth via NFC since API level 16.
83.7\% of devices on the Google Play Store run Android 4.3 or higher.
In addition API level 18 added support for Bluetooth Low Energy (LE), WiFi scan only mode, key store for app-private keys, hardware credential storage and automated UI testing.
All very useful to Tribler.


What alternatives are there to run python code on Android? (QPython, P4A, older projects: PGS4A, SL4A)
Why is P4A chosen: previous work (legacy code), best match

What have I done: cross compiling libraries (dependencies) with python bindings
... etc. (@see "my contributions" in thesis)


Reuse entire existing core
Multi-threading with twisted in python is not perfect: busy reactor thread: API timeouts
Choosen for multi-process: java front-end with python back-end: at least responsive gui with loading indicators



How well does it perform? (per feature / key performance indicator)
Compared to laptop?
Compared to centralized solution? (Youtube)

What advantages does a smart phone have compared to a laptop and desktop?
Bluetooth, NFC, WiFi + p2p, camera, ... etc. all integrated
NFC channel subscription

Reactive programming paradigm on Android

Future work:
comparative analysis
WiFi direct
cutting long running methods in smaller pieces, so multi-threading is smoother (only discovered now, earlier: why is gui hanging?)
release GIL in native C code, mainly for heavy crypto tasks
streaming API for big responses (dumping all known channels from db)



5 min. verhaal, rode lijn
welke keuzes
waarom keuzes zo

functionaliteiten vereist


wat heb ik gedaan, dan pas boek erbij


bijdrage: faciliterend voor future research, nuttig. Tribler features + phone benefits
aantonen? ja, nee, beide is resultaat
significance: demo paper, grand challenge

